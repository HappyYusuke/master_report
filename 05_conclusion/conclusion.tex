\section{まとめ}
本プロジェクトでは、2D-LiDARのデータとYOLOv8を用いた人追従システムの開発を行い、
追従実験と最大追従速度実験により追従性能の検証をした。
本プロジェクトが提案する手法は、2D-LiDARの距離データを俯瞰画像に変換し、
21061枚の俯瞰画像から作成したデータセットとYOLOv8の物体検出アルゴリズム
により両脚部分の検出器を作成した。さらに、動的検出範囲を実装し追従目標の特定をすることで
目標座標を生成し、Happy Eduから目標座標までの距離と角度の偏差を収束させるPID制御により
TurtleBot3 Big Wheelを制御することで、人追従システムを実現した。
結果として、雑多な環境下での追従と0.5[m/s]の最大追従速度が確認でき、
すべての要求仕様を満たすことができた。

\section{今後の課題}
今後の課題として、ロボット台車の最大直進速度が0.5[m/s]であることから、追従対象者が
普段の歩行速度で歩行できない。また、本プロジェクトが提案するシステムは、
人混み中での人追従を想定していないため、ロボットと追従対象者の間に障害物や人が
出現した場合に追従できなくなる可能性がある。これらの課題を解決するため、
人の平均歩行速度以上の移動ができるロボット台車を使用し、追従目標の特定処理を改良
することが必要である。