\section{まとめ}
本研究では、3D-LiDARを用いた人追従システムの開発を行い、歩行者検出性能および追跡・再識別
性能の検証を行った。 本システムでは、Livox社製3D-LiDARの点群特性に最適化を施した
PointPillarsを採用し、独自に構築したデータセットで学習を行うことで高精度な歩行者検出器
を構築した。さらに、ReID3Dを統合することで、オクルージョン(遮蔽)発生後も対象人物を再識別
可能な人追従システムを実現した。 提案手法の有効性を検証するため、独自データセットおよび
公開データセットを用いた評価実験を行った結果、最適化を施したPointPillarsが従来モデルを
大幅に上回る歩行者検出性能を示し、本手法の有用性が確認された。

\section{今後の課題}
今後の課題として、以下の二点が挙げられる。 第一に、歩行者検出性能のさらなる向上である。
本実験の結果より、歩行者が停止し直立している場合に検出漏れが発生する傾向が確認された。
この原因として、学習に使用したデータセットに歩行者が動作している状態の点群が支配的であり、
静止状態のデータが不足していたことが考えられる。したがって、歩行者が停止している状態の
点群を拡充したデータセットを新たに構築し、再学習を行うことで、検出性能の改善が見込まれる。
第二に、追跡および再識別性能の向上である。本システムではトラッキングに線形カルマンフィルタ
を採用しているが、歩行者やロボットの急激な動作により相対軌跡が非線形となる場面では、
追従が困難になる課題が残された。この解決策として、拡張カルマンフィルタ (EKF) や
アンセンテッドカルマンフィルタ (UKF) といった非線形推定手法の導入が有効であると考えられる。
また、再識別モデルとして用いたReID3Dは、交差点監視などを想定しており、対象との距離が遠く
全身が点群に含まれることを前提としている。しかし、人追従タスクではLiDARと歩行者の距離が
近く、画角外れなどにより身体の一部のみしか計測されない状況が頻発する。そのため、近距離
かつ部分的な点群情報からでもロバストに再識別可能な新たなモデルの開発が必要である。