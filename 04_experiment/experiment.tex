\section{実験目的}
本研究では、3D-LiDARを用いた人追従において、障害物による遮蔽(オクルージョン)が発生した
後も、対象を再識別し追跡を継続できるシステムの開発を目的としている。 本システムの有効性を
検証するため、実験ではセンサとしてLivox社製のロボット向け3D-LiDARを採用した。本センサは
独自の走査パターンを有するため、その点群特性に最適化した物体検出ネットワーク
PointPillarsを用い、歩行者検出精度を定量的に評価する。また、公開データセットを用い、
開発したシステムがオクルージョン後の対象人物を正しく再識別できるかについても性能評価を行う。
以上の実験を通じ、3D-LiDARを用いた歩行者検出から再識別に至る一連の性能を明らかにし、構築した
人追従システムの有用性を実証する。

\section{実験方法}
提案システムの有効性を検証するために2種類の実験を行う。第一に、Livox社製3D-LiDARの点群
特性に最適化を施したPointPillarsの歩行者検出精度を定量的に評価する実験である。第二に、構築
した人追従システム全体の追跡性能および再識別性能を評価する実験である。
また、実験機材にはRTX 3070 8GBを搭載したノートPCを使用した。

\subsection{歩行者検出性能に関する比較実験}
本実験では、3.4節で構築した4,500フレームの検証用データセットを用い、PointPillarsの歩行者検出
性能を評価する。評価指標には、物体検出において広く用いられる BEV (Bird’s Eye View) AP 
および 3D AP を採用した。 BEV APは、推定された3次元バウンディングボックスを2次元平面
(鳥瞰図)に投影し、グランドトゥルースとの領域の重複率に基づいて算出される指標である。
一方、3D APは、バウンディングボックスの3次元的な位置および体積の重複率 
(IoU: Intersection over Union) に基づいて算出される。なお、本実験におけるIoUの閾値は、
歩行者検出の一般的な基準に従い 0.5 に設定した。
推論実験の環境構築にあたり、入力データは点群のバイナリファイルをROS 2トピックとして変換
・パブリッシュし、これを検出モデルがサブスクライブする形式を採用した。この際、トピックの
パブリッシュ周波数は 10 Hz に設定している。 また、本手法の有効性を検証するための比較対象
として、既存のオープンソース実装である ros2\_tao\_pointpillars パッケージの標準
モデルを用いた。

\subsection{追跡・再識別性能に関する比較実験}
本実験では、公開データセットを用い、構築した人追従システム全体の追跡性能および再識別性能
を評価する。 評価用データセットには、TPT-BENCH
\cite{TPT-Bench: A Large-Scale, Long-Term and Robot-Egocentric Dataset for 
Benchmarking Target Person Tracking} を採用した。ロボット知覚に関する既存のデータセット
としては、RoboSense\cite{RobSense: A Robust Multi-modal Foundation Model for 
Remote Sensing with Static, Temporal, and Incomplete Data Adaptability} や JRDB
\cite{JRDB-PanoTrack: An Open-world Panoptic Segmentation and Tracking Robotic 
Dataset in Crowded Human Environments} などが挙げられる。しかし、本研究の主眼である
「特定の人物を継続的に追跡するタスク (Target Person Tracking: TPT)」における評価には、
TPTに特化したデータセットが最適であるためTPT-BENCHを選定した。 TPT-BENCHは、LiDAR点群、
RGB画像、深度情報の3種類のセンサ情報を含んでおり、屋内・屋外、あるいは日中・夜間といった
多様な環境下で収集された大規模データセットである。

評価指標には、Average Overlap (AO)、F1-score、および Avg Max Recall (AMR) の3つを用
いる。 AOは、推定されたバウンディングボックスとグランドトゥルース(正解データ)との領域
重複度(IoU)の平均値を示す指標である。F1-scoreは、追跡の適合率 (Precision) と再現率 
(Recall) の調和平均であり、検出の正確さと網羅性のバランスを評価する。AMRは、ターゲットを
どの程度の割合で正しく捉え続けられたかを示す、追跡の頑健性を測る指標である。



\section{実験結果}
\subsection{歩行者検出性能に関する比較実験}
歩行者検出の実験結果をTable \ref{Evaluation results of pedestrian detection.}に示す。
ros2\_tao\_pointpillarsのBEV APは35.47[\%]、3D APは18.50[\%]であったのに対し、
本手法はBEV APが79.86[\%]、3D APが61.59[\%]であり、いずれの指標においても従来のモデル
を大幅に上回る結果となった。どちらのモデルも誤検出があり、ros2\_tao\_pointpillarsでは、
2人の歩行者を1つのバウンディングボックスとして検出するケースが見られた。一方、本手法では、
概ね歩行者を検出できていたが、歩行者が停止し、直立している場合に検出されない現象が見られた。

\begin{table}[h]
  \centering
  \caption{Evaluation results of pedestrian detection.}
  \label{Evaluation results of pedestrian detection.}
  \begin{tabular}{ccc} \toprule
    Model & BEV AP (\%) & 3D AP (\%) \\ \midrule
    ros2\_tao\_pointpillars & 35.47 & 18.50 \\
    Ours & 79.86 & 61.59 \\ \bottomrule
  \end{tabular}
\end{table}


\section{考察}
実験結果から、Livox社製3D-LiDARの点群特性に最適化を施したPointPillarsが、
従来のモデルを大幅に上回る歩行者検出性能を示したことが確認できた。これは、
本研究で提案した独自のデータセットを用いた学習が、センサ特性に適応した特徴抽出を可能にし、
歩行者検出においてドメインギャップを効果的に克服したためであると考えられる。
ただし、誤検出の傾向には両モデルで違いが見られた。ros2\_tao\_pointpillarsでは、
近接する歩行者を1つのバウンディングボックスとして誤検出するケースが見られた。
これは、公開されている学習済みモデルでは、2人の歩行者を1つのバウンディングボックスで
表現するようなアノテーションが存在したことが影響している可能性がある。
一方、本手法では、歩行者が停止し直立している場合に検出されない現象が見られた。これは、
データセットにおいて、歩行者が動いている状態の点群が多く含まれていたことが影響している
と考えられる。