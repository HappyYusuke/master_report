\section{はじめに}
近年の日本において、少子高齢社会による人手不足が課題となっている。2023年の65歳以上の
人口は3623万人であり、総人口に占める65歳以上の割合(以下、高齢化率)は29.1[\%]と過去最高である\cite{統計からみた我が国の高齢者}。
また、2070年での高齢化率は38.7[\%]に達し、2.6人に1人が65歳以上であると推計されている\cite{高齢化の現状と将来像}。
加速する少子高齢化により、就業者不足の問題が深刻化しており、解決策の1つとしてロボットによる
作業の自動化やサポートの導入が増えている。建設業では、2D-LiDARを用いた自動追従台車である
「かもーん」が建設現場で導入されており、運用実績を上げ続けている\cite{近年の建設工事用ロボット開発について}。
また、製造業では2D-LiDARを用いた協働運搬ロボットである「サウザー」が実用化されており、
「自動追従走行機能」によって運搬業務をサウザーで自動化しており、製造業界だけでなく空港や市役所
などの公共環境における導入例があり、世界各地で約400台の販売実績がある\cite{既存AGVを超える特長を持った協働運搬ロボット「サウザー」}。
以上のことから、2D-LiDARを用いた人追従ロボットへの需要と期待は増加し続けていることがわかる。\\ \indent
2D-LiDARを用いた人追従ロボットに関する手法には、深層学習を用いる手法
\cite{深層学習を用いた人追従機能の開発}
\cite{People Detection and Tracking Using LIDAR Sensors}
\cite{Temporal convolutional networks for multi-person activity recognition using a 2D LIDAR}
\cite{Tracking People in a Mobile Robot From 2D LIDAR Scans Using Full Convolutional Neural Networks for Security in Cluttered Environments}
、背景減算によって人の両脚部分を検出する手法
\cite{Tracking People Using Ankle-Level 2D LiDAR for Gait Analysis}
、AOAタグを2D-LiDARと組み合わせる手法
\cite{A Robust Autonomous Following Method for Mobile Robots in Dynamic Environments}
などがある。これらの手法では、主に距離データをもとに人の両脚部分を検出するが、
雑多な環境下では追従性能が低下する可能性がある。また、実験環境に椅子や机などのオブジェクトが
なく、広域な経路での実験により追従性能を検証していることから、
雑多な環境下での追従性能が評価されていない。
2D-LiDARから提供される距離データでは、椅子や机などの
脚部分が人の両脚部分と類似しているため、誤検出してしまう課題があり、これに伴い追従速度が低下する課題もある。\\ \indent
本プロジェクトでは、屋内環境による雑多な環境下で追従でき、ロボットの最大直進速度で追従できる
人追従システムを開発する。

\section{論文構成}
本レポートの構成について述べる。第2章では、これまでの2D-LiDARを用いた人追従に関連する
従来研究について述べる。第3章では、本プロジェクトでの提案手法を述べる。
第4章では、提案手法による人追従能力の検証結果について述べる。
第5章では本プロジェクトをまとめ、結論及び今後の課題について述べる。