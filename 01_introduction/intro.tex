\section{はじめに}
近年,物流,医療,警備といった多様な分野において,労働力不足の解消や業務効率化を
目的とした自律移動ロボットの社会実装が強く期待されている.こうしたロボットが人と
共存し,協調してタスクを遂行する上で,特定の人物を認識し追従する「人追従機能」は
極めて重要な要素技術である.しかし,動的に環境が変化する実環境下においては,他の
歩行者の横切りや障害物による遮蔽(オクルージョン)が発生し,追従対象が一時的にセ
ンサの視野から消失する事態が頻発する.追従の継続性を担保するためには,オクルージ
ョンからの復帰時に,再出現した人物が追従対象本人であるかを正しく判別する「再識別
(Re-Identification)」が不可欠となる.

実環境における人追従システムの実現に向けては,これまで多角的なアプローチが試みら
れてきた.カメラ(RGB画像)を用いた手法では,深層学習による人物検出と追跡アルゴ
リズムを組み合わせたシステムが広く研究されている.これらはテクスチャ情報が豊富な
ため,良好な照明条件下では高い識別性能を発揮する.しかし,服装などの外観情報や環
境照度に強く依存するため,逆光や低照度といった照明条件の変動により認識精度が著し
く低下するという課題がある. 一方,2D-LiDARを用いた手法では,環境の明るさに依存
しないロバストな計測が可能であるが,取得できる情報は水平断面の距離データに限定さ
れる.一般に,脚部や胴体の断面をクラスタリングして追跡を行うが,水平断面形状のみ
から個人の身体的特徴量を抽出することは困難であり,他者との識別性は低い.センサの
種類にかかわらず,人混みや障害物の多い環境ではオクルージョンが不可避であり,カル
マンフィルタ等の運動モデルに基づく追跡アルゴリズムだけでは,長時間の遮蔽や不規則
な移動によって対象を見失った後の復帰が困難となる.したがって,対象再出現時に周囲
の人物と明確に区別し,追従を再開するためのロバストな再識別手法の確立が,人追従技
術における喫緊の課題となっている.

そこで本研究では,3D-LiDARを用いた人追従システムにおいて,再識別モデルReID3D[1]を
導入し,オクルージョンに対する高いロバスト性を有するシステムの構築を目的とする.
本研究の新規性は,追従対象の「検出・追跡」に留まらず,3次元点群情報に基づいた「再
識別」に焦点を当てた点にある.従来の3D-LiDARを用いた人追従研究の多くは,高精度な
検出や運動予測に主眼を置いており,点群を用いた個人の再識別に着目した事例は少ない.
本研究で採用するReID3Dは,Guoらによって提案された人物再同定フレームワークであり,
人物の3D点群から身長,体型,歩容といった幾何学的特徴を高次元ベクトルとして抽出可能
である.これを人追従システムに応用することで,照明条件に左右されず,かつオクルージ
ョン後であっても対象を正確に再発見可能なシステムの実現を目指す.

\section{論文構成}
本レポートの構成について述べる。第2章では、これまでの2D-LiDARを用いた人追従に関連する
従来研究について述べる。第3章では、本プロジェクトでの提案手法を述べる。
第4章では、提案手法による人追従能力の検証結果について述べる。
第5章では本プロジェクトをまとめ、結論及び今後の課題について述べる。