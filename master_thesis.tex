\documentclass[12pt,a4paper]{jreport}
\usepackage{KIT-j-master}
\usepackage{latexsym}
\usepackage{amsmath}
\usepackage{amssymb}
\usepackage{comment}
%\usepackage{graphicx}
%\usepackage{subeqn}
%\usepackage{eqnarray}
\usepackage{newtheorem_thesis}
\usepackage[dvipdfmx]{color}
\usepackage[dvipdfmx]{graphicx}
\usepackage{subfigure}
\usepackage{slashbox}
\usepackage{here}
\usepackage{cite}
%KANAZAWA custom
\usepackage[hang,small,bf]{caption}
%\usepackage[subrefformat=parens]{subcaption}
\captionsetup{compatibility=false}
\usepackage{indentfirst}

%%%%%%%%%%%%%%%%%%%%%%%%%%%%%%%%%%%%%%%%%%%%%%%%%%%%%%%%%%%%%%%%%%%%%%%%%
%ソースコードを表示するための設定
%(テンプレートには無い設定・listings & jlistingをインストールすること)
%%%%%%%%%%%%%%%%%%%%%%%%%%%%%%%%%%%%%%%%%%%%%%%%%%%%%%%%%%%%%%%%%%%%%%%%%
\usepackage{ascmac}
\usepackage{here}
\usepackage{txfonts}
\usepackage{listings,jlisting}
\usepackage{bm}
\usepackage{lscape}

\lstdefinestyle{customC++}{
	language={C++},% 使用する言語
	tabsize=2,%
	basicstyle=\footnotesize,%
	commentstyle=\textit,%
	classoffset=1,%
	keywordstyle=\bfseries,%
	frame=tRBl,framesep=5pt,%
	showstringspaces=false,%
	numbers=left,stepnumber=1,numberstyle=\footnotesize,%
	breaklines=true,%        折り返しの設定
	breakatwhitespace=true,% 折り返しの設定
	lineskip=-0.3ex%  
}

\lstdefinestyle{customhtml}{
	language={html},% 使用する言語
	tabsize=2,%
	basicstyle=\footnotesize,%
	commentstyle=\textit,%
	classoffset=1,%
	keywordstyle=\bfseries,%
	frame=tRBl,framesep=5pt,%
	showstringspaces=false,%
	numbers=left,stepnumber=1,numberstyle=\footnotesize,%
	breaklines=true,%        折り返しの設定
	breakatwhitespace=true,% 折り返しの設定
	lineskip=-0.3ex%  
}

\lstdefinestyle{custompy}{
	language={python},% 使用する言語
  	basicstyle=\ttfamily\scriptsize,
 	commentstyle={\itshape \color[cmyk]{1,0.4,1,0}},
	classoffset=1,
	keywordstyle={\bfseries \color[cmyk]{0,1,0,0}},
	stringstyle={\ttfamily \color[rgb]{0,0,1}},
	frame=tRBl,
	framesep=5pt,
	showstringspaces=false,
	numbers=left,
	stepnumber=1,
	numberstyle=\tiny,
	tabsize=2,
}
\lstset{escapechar=@,style=customC++}

%\renewcommand{\lstlistingname}{リスト}
%\lstset{language=C++,%
%tabsize=2,%
%basicstyle=\footnotesize,%
%commentstyle=\textit,%
%classoffset=1,%
%keywordstyle=\bfseries,%
%frame=tRBl,framesep=5pt,%
%showstringspaces=false,%
%numbers=left,stepnumber=1,numberstyle=\footnotesize,%
%breaklines=true,%        折り返しの設定
%breakatwhitespace=true,% 折り返しの設定
%lineskip=-0.3ex%         行間をつめる設定
%}

\title{Development of a Human Following System Using 3D LiDAR and ReID3D}
\author{Yusuke Kanazawa}
\eauthor{Yusuke Kanazawa}
\school{Kanazawa Institute of Technology}
\major{Graduate School of Engineering, }
\course{Mechanical Engineering}
\adviser{Kosei Demura}
\date{Academic Year 2025\\Janualy 26, 2026}

% --- 目次のChapter番号幅 強制修正コード (ここから) ---
\makeatletter
\renewcommand{\l@chapter}[2]{%
  \ifnum \c@tocdepth >\m@ne
    \addpenalty{-\@highpenalty}%
    \vskip 1.0em \@plus\p@
    \begingroup
      \parindent \z@ \rightskip \@pnumwidth
      \parfillskip -\@pnumwidth
      \leavevmode \bfseries
      % 【ここが変更点】
      % tempdimaではなく、numberlineというコマンドの中身を直接書き換えて
      % 強制的に「10文字分(10.0em)」の幅を確保させます。
      \renewcommand{\numberline}[1]{\makebox[5.0em][l]{##1}} 
      #1\nobreak\hfil \nobreak\hb@xt@\@pnumwidth{\hss #2}\par
      \penalty\@highpenalty
    \endgroup
  \fi
}
\makeatother
% --- 目次のChapter番号幅 強制修正コード (ここまで) ---

\begin{document}
\maketitle

%\pagestyle{headings}
\pagenumbering{roman}
\tableofcontents
\listoffigures
\listoftables
\newpage
\pagenumbering{arabic}
\setcounter{page}{1}

%%%%%%%%%%%%%%%%%%%%%%%%%%%%%%%%%%%%%%%%%%%%%%%%%%%%%%%%%%%%%%%%%%%%%%%%%
\chapter{Introduction}
\section{はじめに}
近年,物流,医療,警備といった多様な分野において,労働力不足の解消や業務効率化を
目的とした自律移動ロボットの社会実装が強く期待されている.こうしたロボットが人と
共存し,協調してタスクを遂行する上で,特定の人物を認識し追従する「人追従機能」は
極めて重要な要素技術である.しかし,動的に環境が変化する実環境下においては,他の
歩行者の横切りや障害物による遮蔽(オクルージョン)が発生し,追従対象が一時的にセ
ンサの視野から消失する事態が頻発する.追従の継続性を担保するためには,オクルージ
ョンからの復帰時に,再出現した人物が追従対象本人であるかを正しく判別する「再識別
(Re-Identification)」が不可欠となる.

実環境における人追従システムの実現に向けては,これまで多角的なアプローチが試みら
れてきた.カメラ(RGB画像)を用いた手法では,深層学習による人物検出と追跡アルゴ
リズムを組み合わせたシステムが広く研究されている.これらはテクスチャ情報が豊富な
ため,良好な照明条件下では高い識別性能を発揮する.しかし,服装などの外観情報や環
境照度に強く依存するため,逆光や低照度といった照明条件の変動により認識精度が著し
く低下するという課題がある. 一方,2D-LiDARを用いた手法では,環境の明るさに依存
しないロバストな計測が可能であるが,取得できる情報は水平断面の距離データに限定さ
れる.一般に,脚部や胴体の断面をクラスタリングして追跡を行うが,水平断面形状のみ
から個人の身体的特徴量を抽出することは困難であり,他者との識別性は低い.センサの
種類にかかわらず,人混みや障害物の多い環境ではオクルージョンが不可避であり,カル
マンフィルタ等の運動モデルに基づく追跡アルゴリズムだけでは,長時間の遮蔽や不規則
な移動によって対象を見失った後の復帰が困難となる.したがって,対象再出現時に周囲
の人物と明確に区別し,追従を再開するためのロバストな再識別手法の確立が,人追従技
術における喫緊の課題となっている.

そこで本研究では,3D-LiDARを用いた人追従システムにおいて,再識別モデルReID3D[1]を
導入し,オクルージョンに対する高いロバスト性を有するシステムの構築を目的とする.
本研究の新規性は,追従対象の「検出・追跡」に留まらず,3次元点群情報に基づいた「再
識別」に焦点を当てた点にある.従来の3D-LiDARを用いた人追従研究の多くは,高精度な
検出や運動予測に主眼を置いており,点群を用いた個人の再識別に着目した事例は少ない.
本研究で採用するReID3Dは,Guoらによって提案された人物再同定フレームワークであり,
人物の3D点群から身長,体型,歩容といった幾何学的特徴を高次元ベクトルとして抽出可能
である.これを人追従システムに応用することで,照明条件に左右されず,かつオクルージ
ョン後であっても対象を正確に再発見可能なシステムの実現を目指す.

\section{論文構成}
本レポートの構成について述べる。第2章では、これまでの2D-LiDARを用いた人追従に関連する
従来研究について述べる。第3章では、本プロジェクトでの提案手法を述べる。
第4章では、提案手法による人追従能力の検証結果について述べる。
第5章では本プロジェクトをまとめ、結論及び今後の課題について述べる。
%%%%%%%%%%%%%%%%%%%%%%%%%%%%%%%%%%%%%%%%%%%%%%%%%%%%%%%%%%%%%%%%%%%%%%%%%
\chapter{Related Work}
\section{2D LiDARを用いた人検出手法}
2D LiDARを用いた研究では,取得した水平断面の点群データをクラスタリングし,脚部や
胴体の断面形状として検出した後,カルマンフィルタ等を用いて時系列的な追跡を行う
手法が多い.
Arrasら\cite{Using Boosted Features for the Detection of People in 2D Range Data}は
、2Dレーザーレンジファインダ(LRF)から得られる平面スキャンデータを用いて、
複雑な環境下でもロバストに人物を検出する手法を提案している。従来の多くのアプ
ローチは、人の脚を検出・追跡するために事前に定義された特徴量を使用していたが、
これらは手動での設計や調整に依存しているという課題があった。
これに対しArrasらは、多数の特徴量から有効なものを教師あり学習によって選択・統合
するアプローチにより、環境に適応した高精度な検出器を構築している。

提案手法では、まずLRFから取得された点群データに対し、隣接するビーム間の距離が一定
以上離れた箇所で分割する「ジャンプディスタンス条件」を用いてセグメント化を行う。
得られた各セグメントを脚の候補とし、その形状や統計的性質を表す14種類のスカラー特徴
量を算出している。具体的には、セグメントを構成する点数、標準偏差、外れ値に強い
指標である中央値からの平均偏差、隣接セグメントへのジャンプ距離、セグメント幅、
最小二乗法による直線および円への適合度(直線性・円形度)、適合円の半径、境界長、
境界の規則性、平均曲率、境界の凸性を測る平均角度差といった静的な幾何学的特徴に
加え、連続するスキャン間の差分から算出される平均速度という動的特徴が定義されている。

学習アルゴリズムにはAdaBoostを採用し、これらの単純な特徴量(弱い識別器)を組み合わ
せることで、人の脚に対応するセグメントとそうでないものを判別する強い識別器を構築
している。AdaBoostを用いることで、識別性能を最大化するように、最も情報量の多い
特徴量と最適な閾値をデータから自動的に学習することが可能となる。

実験は、静止した人物や移動する人物が存在する廊下や散らかったオフィス環境に
おいて実施された。学習された識別器の性能評価の結果、家具や物体が多く存在する複雑
な環境下においても、90\%を超える高い正解率で人物を検出できることが示されている。
特に、従来研究で頻繁に用いられる幾何学的ルール(脚の幅やジャンプ距離に固定閾値を
設ける手法など)に基づくヒューリスティックな手法との比較において、提案手法は誤検出
率を大幅に低減させ、その優位性を実証している。また、ある環境で学習させた識別器を、
学習データに含まれていない未知の環境に適用した場合でも高い精度が維持されており、
学習されたモデルが高い汎用性を有していることが確認された。

さらに、学習された識別器の分析を通じて、人検出に有効な特徴量に関する重要な知見が得
られている。AdaBoostによって選択された特徴量の重みを分析した結果、最も識別に寄与
していたのは、単純な脚の幅や円形度ではなく、フィッティングされた円の「半径」および
セグメントの「凸性(平均角度差)」であった。これは、2Dスキャンデータにおける脚の
形状が、必ずしも綺麗な円弧や直線ではなく多様な形状をとる中で、サイズ感と外側への
膨らみ具合が最もロバストな指標であることを示唆している。加えて、動的特徴である
「移動速度」を追加した場合の実験結果から、静的な幾何学的特徴のみを用いた場合と比較
して精度の向上はわずかであることが示された。このことからArrasらは、静止している
人物も含めたロバストな検出を行う上では、移動情報は必須ではなく、適切な幾何学的特徴
の組み合わせだけで十分に高精度な検出が可能であると結論付けているが、オクルージョン
後の再検出や追従復帰については言及されていない。



\section{3D LiDARを用いた人追従手法}
3D LiDARを用いた研究では,3次元空間の点群データから地面点群を除去し,ユークリッド
クラスタリング等を用いて人物候補となる点群クラスタを抽出した後,カルマンフィルタや
パーティクルフィルタを用いて追跡するパイプラインが多い。Yanら
\cite{Online learning for 3D LiDAR-based human detection: experimental analysis 
of point cloud clustering and classification methods}
は、3D LiDARを搭載した移動ロボットが、走行環境に適応しながら人物検出器をオンライン
で学習するフレームワークを提案している。3D LiDARを用いた人物検出において、距離に
よる点群密度の変化や、学習データの作成にかかる人的コスト(アノテーションの手間)は
大きな課題であった。これに対しYanらは、少数のラベル付きデータで初期化した識別器
(SVM)を、ロボットの稼働中に得られるデータを用いて自動的に再学習させる手法を開発
した。

提案手法の核となるのは、適応的なクラスタリング手法と「P-N Experts」と呼ばれる教師
データ生成メカニズムである。まず、点群のセグメンテーション(クラスタリング)において、
LiDARからの距離に応じて疎になる点群密度に対応するため、センサを中心とした同心円状の
領域ごとに異なる距離閾値を設定する適応的クラスタリングを採用している。これにより、
近距離の密な点群と遠距離の疎な点群の双方において、ユークリッド距離に基づく従来手法
や深度画像ベースの手法よりも高精度な切り出しを実現している。

人物識別においては、幾何学的な特徴量を用いたSVM分類器が採用されているが、特に遠距
離における検出性能を向上させるため、点群を高さ方向に10個のスライスに分割し、各スラ
イスの幅や奥行きを記述する「スライス特徴量(Slice feature)」を新たに導入している。
これにより、点数が少ない遠方の人物に対してもロバストな特徴抽出が可能となっている。

オンライン学習のプロセスでは、マルチターゲット追跡(UKF)の結果を利用して、識別器
の誤りを自動的に修正する。具体的には、「P-expert」が追跡軌跡の連続性を利用して検出
漏れ(False Negatives)を拾い上げて正例として追加し、「N-expert」が静止物体などの
誤検出(False Positives)を特定して負例として追加する。このようにして自動生成され
た学習データを用いて識別器を定期的に再学習させることで、人手による追加のアノテーシ
ョンなしに、環境特有の人物や背景の特徴に適応していくことが可能である。

大規模な屋内環境で取得されたデータセット(L-CAS 3D Point Cloud People Dataset)を
用いた評価実験の結果、提案された適応的クラスタリングは従来手法と比較して高いセグメ
ンテーション精度を示し、また、オンライン学習によって更新された識別器は、オフライン
で手動学習された識別器よりも高いF値(精度と再現率の調和平均)を達成することが確認
されている。Yanらの研究は、3D LiDARを用いた人物検出において、幾何学的特徴量の有効性
と、追跡情報をフィードバックループに組み込むことによる自己教師あり学習の有用性を
示した重要な成果であるが、オクルージョン後の再検出や識別については言及されていない。



\section{単眼カメラを用いたオクルージョンを考慮した人追従手法}
カメラ(RGB画像)を用いた人追従システムは、対象の色やテクスチャといった豊富な光学
的情報(Photometric features)を活用できる点が特徴である.初期の研究では,色ヒス
トグラムやHOG (Histogram of Oriented Gradients) 特徴量を用いた検出手法が主流であ
ったが,近年では深層学習(Deep Learning)と追跡アルゴリズムを組み合わせた手法が広
く研究されている.また,事前学習済みモデルをそのまま用いるのではなく,環境や対象
に応じて適応するオンライン学習を取り入れたアプローチも提案されている.Koideら
\cite{Monocular Person Tracking and Identification with On-line Deep Feature 
Selection for Person Following Robots}
は、LiDARやRGB-Dカメラと比較して安価で導入が容易な「単眼カメラ」のみを用いた移動
ロボットの人追従システムを提案している。

単眼カメラを用いた人追従においては、主に二つの技術的課題が存在する。第一の課題は、
「奥行き情報が得られない画像から、いかに対象の3次元位置を正確に推定するか」という
点である。第二の課題は、「計算リソースの限られたロボット上で、照明変動や姿勢変化に
頑健な追跡をリアルタイムで実現するか」という点である。従来、計算コストの低いHOG
特徴量やカラーヒストグラムを用いた手法(KCFなど)は高速であるが、照明変化や背景と
の同化に脆弱であった。一方で、深層学習(CNN)を用いた手法は高い識別能力を持つもの
の計算負荷が極めて高く、組み込みGPUを搭載したモバイルロボットでのリアルタイム動作
は困難であった。

これらの課題に対し、Koideらは以下の二つのアプローチを統合することで解決を図ってい
る。

第一に、3次元位置推定の課題に対しては、深層学習ベースの骨格検出(OpenPose)と幾何
学的拘束を組み合わせた追跡フレームワークを導入している。具体的には、画像上で検出さ
れた人物の「足首」と「首」の位置に着目し、これらが地平面(Ground Plane)上に存在す
るという仮定と、カルマンフィルタ(UKF)による身長推定を組み合わせることで、距離セ
ンサを用いずに高精度なロボット座標系での位置推定を実現した。この手法は、足元が障害
物で隠れた場合(オクルージョン)でも、首の位置情報を用いて追跡を継続できる利点を
持つ。

第二に、計算コストと堅牢性のトレードオフに対しては、「オンライン深層特徴選択
(ODFS: On-line Deep Feature Selection)」という手法を提案している。これは、
ImageNetで事前学習されたVGG-16モデルから得られる膨大な特徴マップ(Convolutional 
Channel Features: CCF)を全て使用するのではなく、現在のフレームにおいて
「ターゲット」と「背景」を分離するのに最も有効な特徴量だけをオンラインブーステ
ィングによって動的に選択する手法である。これにより、照明条件が変化した場合には色
特徴を、形状が特徴的な場合にはエッジ特徴を重点的に利用するといった適応が可能とな
り、深層学習の表現力を活かしつつ、組み込みコンピュータ(NVIDIA Jetson TX2)上で平
均20fpsというリアルタイム処理を達成している。

評価実験では、屋内および屋外環境において、照明変化、複雑な背景、ターゲットの回転や
遮蔽が発生するシナリオで検証が行われた。その結果、提案手法はKCFやTLDといった既存の
トラッカーと比較して最も高い追跡成功率と精度を記録し、単眼カメラという限られたセン
サ構成であっても、深層学習の効率的な利用により実環境での頑健な人追従が可能であるこ
とを実証している。一方で、単眼カメラを用いる本手法は、原理的に奥行き情報の直接計測
が困難であり、かつ視野角も限定されるという課題を残している。実験においても、特定の
条件下では高い追従性能を示したものの、対象がカメラに接近しすぎた際の検出失敗や、遠
距離における距離推定精度の低下といった、センサ特性に起因する誤推定や検出不可の条件
が存在することが報告されている。

\section{PointPillars}
Langら\cite{PointPillars: Fast Encoders for Object Detection from Point Clouds}は、
自動運転やロボティクスにおいて極めて重要となる3D物体検出タスクにおいて、推論速度
と検出精度の両立を目指した新たなネットワークアーキテクチャ「PointPillars」を提案
している。

従来、点群データ(Point Cloud)を用いた3D物体検出には、点群をボクセル(Voxel)化
して3D畳み込みニューラルネットワーク(3D CNN)を適用する手法や、点群
を鳥瞰図(Bird's Eye View: BEV)などに投影して2D CNNを適用する手法が存在した。
前者は高い精度を誇る一方で計算コストが非常に高く(例:VoxelNet)、後者は高速である
ものの、3次元形状の情報を損失しやすく精度が劣るという課題があった。また、PointNet
のように点群を直接処理する手法は計算効率が良いが、大規模な点群に対するスケーラビ
リティに課題があった。これに対し、PointPillarsは、点群を垂直方向の柱(Pillar)状
に整理して処理する新しいエンコーダを採用することで、3D畳み込みを一切使用せず、標
準的な2D畳み込みのみで構成される高速かつ高精度な検出パイプラインを実現している。

アーキテクチャは、大きく分けてPillar Feature Net (PFN)、Backbone (2D CNN)、
Detection Head (SSD) の3つの部分から構成される。第一に、Pillar Feature Net (PFN)
は、生の点群データを「擬似画像(Pseudo-image)」に変換する役割を担う。まず、入力
された点群は$xy$平面上のグリッドに基づいて、垂直方向に伸びる「ピラー(Pillar)」
に分割される。各ピラー内の点は、座標 $(x, y, z)$、反射強度 $r$、幾何学的中心から
のオフセット $x_c, y_c, z_c$、ピラー中心からのオフセット $x_p, y_p$ を含む9次元
の特徴ベクトルとして表現される。これに対し、簡略化されたPointNet(線形層、Batch
Norm、ReLU)を適用することで、各点の特徴を高次元空間へ写像し、さらにピラーごとの
最大プーリング(Max Pooling)を行うことで、各ピラーを代表する特徴ベクトルを抽出
する。この処理により、点群データは $(C, H, W)$ の形式を持つ2次元の擬似画像へと
変換される。この手法は、従来のボクセルベースの手法とは異なり、垂直方向のビン(bin)
分割を行わないため、ハイパーパラメータの調整が容易であり、かつスパースなデータ
構造を効率的に扱えるという利点を持つ。第二に、Backboneネットワークは、生成された
擬似画像を入力とし、2D CNNを用いて高レベルな特徴抽出を行う。このバックボーンは、
特徴マップの解像度を下げるダウンサンプリングブロックと、それらを元の解像度に合わ
せてアップサンプリングして結合するブロックから構成されており、異なる受容野
(Receptive Field)を持つ特徴を統合することで、様々なサイズの物体に対応可能な表現
を獲得している。第三に、Detection Headには、物体検出で広く用いられているSingle 
Shot Detector (SSD) が採用されている。これにより、バックボーンからの出力特徴
マップに基づき、3Dバウンディングボックスの回帰(位置、サイズ、角度)およびクラス
分類を行う。

評価実験において、PointPillarsはKITTI 3D物体検出ベンチマークにおいて、従来の
最高精度を誇る手法(MV3D, VoxelNet, SECONDなど)を凌駕する性能(BEVおよび3D検出
の両方において)を示した。特筆すべきは、その推論速度であり、デスクトップGPU上で
62 Hzという極めて高速な動作を実現している。これは、3D畳み込みを排除し、最適化さ
れた2D畳み込み演算のみを使用したことによる成果である。さらに、固定エンコーダ
(手動設計された特徴量)を使用する従来手法と比較しても、学習可能なエンコーダ
(Learned Encoder)を使用することで、速度を犠牲にすることなく大幅な精度向上を達成
していることが実証されている。

Langらによって提案されたPointPillars は、推論速度と検出精度のバランスにおいて優れ
た手法であるが、その構造上の特性に起因するいくつかの技術的課題も明らかになっている。
まず、類似した幾何学的特徴を持つ対象の識別において誤検知が生じやすいという点が挙げ
られる。PointPillarsは点群を垂直方向のピラーに圧縮して特徴抽出を行うため、高さ方向
の詳細情報が抽象化される過程で、ポールや木の幹といった細長い垂直構造物を歩行者とし
て誤認する事例や、歩行者とサイクリストを互いに誤分類する事例が報告されている。また
、乗用車(Car)の検出においても、バンやトラムといった類似形状の車両クラスを誤って
検出するケース(False Positive)が確認されている。
次に、観測条件が悪い状況下での検出能力の限界である。他のLiDARベースの手法と同様に、
対象が他の物体によって部分的に遮蔽(オクルージョン)されている場合や、センサから
遠く離れており点群密度が極端に低い場合において、検出漏れ(False Negative)が発生
しやすい傾向がある。さらに、空間解像度と処理速度の間にトレードオフが存在する点も
課題である。推論速度を優先してグリッドサイズを大きく設定すると、車両のような大きな
物体の検出性能は維持されるものの、歩行者やサイクリストといった小規模な対象の検出
精度が低下することがアブレーションスタディによって示されている。したがって、実環境
で多様な対象を同時にかつ高速に検出するためには、このトレードオフを慎重に調整する
必要がある。



\section{ReID3D}
Guoらは、公共のセキュリティや監視システムにおいて重要な役割を担う人物再同定
(Person Re-identification: ReID)タスクにおいて、従来のカメラベースの手法が抱える
環境依存性の問題を克服するため、LiDARセンサーのみを用いた新たなフレームワーク
「ReID3D」を提案している。 RGBカメラを用いた従来のReID手法には、服装の色や
テクスチャといった外見情報(Appearance)に強く依存しているという根本的な課題がある。
そのため、夜間や低照度環境では視覚情報が欠落し、また複雑な背景においては視覚的な
曖昧さが増大するため、識別精度が著しく低下してしまう。 この問題に対し、深度情報を
利用するアプローチとしてKinectやミリ波レーダーを用いる研究も存在するが、Kinectは
屋内利用に限られ計測範囲が狭く、レーダーは解像度が低いため人物の詳細な識別が困難
であるという制約が存在した。

これらの課題を解決するため、Guoらは、照明条件に影響されず、かつ長距離でも高精度な
3D構造情報を取得可能なLiDARに着目した。彼らは、LiDAR点群から、身長や体型、歩行
(Gait)といった本質的な特徴(Intrinsic Features)を抽出することで、服装の変化や
暗闇にも頑健なReIDの実現を目指した。しかし、LiDARを用いたReIDの研究は前例がなく、
学習に必要なデータセットが存在しないことが大きな障壁となっていた。 このデータ不足
を解決するため、著者らは初の実環境LiDAR ReIDデータセット「LReID」およびシミュレー
ションデータセット「LReID-sync」を構築した。LReIDは屋外の多様な環境下で収集された
320名のIDを含む実データであり、LReID-syncはUnity3Dを用いて生成された600名の合成
データである。

本手法の中核となるのは、点群データの希薄性(Sparsity)と単一視点による情報の欠落を
補うための「マルチタスク事前学習(Multi-task Pre-training)」戦略である。
まずLReID-syncを用いてエンコーダを学習させる際、欠損した点群から完全な全体形状
を復元する「点群補完(Point Cloud Completion)」と、人体モデルのパラメータを推定
する「SMPLパラメータ学習」という二つのタスクを課すことで、エンコーダに人体の
幾何学的構造を事前に獲得させ、実データでの学習を効果的に支援している。また、
ReIDネットワークには、新たに設計された「Graph-based Complementary Enhancement 
Encoder (GCEE)」が採用された。GCEEは、特徴空間上で動的にグラフを構築するGCNを
バックボーンとし、「Eraser」モジュールを持つComplementary Feature Extractor (CFE) 
を導入している。CFEは、あるフレームから抽出した主要な特徴を次のフレームから消去
(Erase)して処理することで、フレーム間の情報の重複を排除し、より包括的で識別性
の高い特徴抽出を可能にしている。さらに、時系列情報の統合にはTransformerが用
いられ、歩行動作などの動的特徴を効率的に集約している。

評価実験において、ReID3Dは構築されたLReIDデータセット上で、カメラベースの最先端手
法(TCLNetなど)と比較して圧倒的なロバスト性を示した。特に低照度環境下においては、
カメラベース手法の精度が大幅に低下する中、ReID3DはRank-1精度で93.3\%
(全体では94.0\%)という高い性能を達成し、照明条件に左右されないLiDARの優位性を
実証した。 結論として、ReID3Dは、シミュレーションデータを活用した事前学習と、
点群の特性に特化したGCEEを組み合わせることで、LiDAR点群から個人の本質的な3D特徴を
抽出することに成功し、従来のカメラだけでは困難であった悪条件下での人物再同定という
課題を解決した先駆的な研究であるといえる。

ReID3Dは、低照度環境や幾何学的構造の取得において優れた性能を示す一方で、十分な照度
が確保された明所(Normal light)環境においては、ビデオベースの最先端手法と比較して
認識精度が劣るという課題が確認されている。 これは、ビデオベースの手法が、明るい環
境下で得られるリッチな外見情報(服装の色、柄、テクスチャなど)を最大限に活用できる
のに対し、LiDARベースの手法であるReID3Dは、形状情報と反射強度のみに依存し、色彩情
報を利用できないことに起因している。また、特徴空間の可視化分析において、ReID3Dは、
カメラベースの手法であれば容易に識別可能な特定の歩行者の識別に失敗するケースも報告
されている。これらの結果は、LiDARとカメラがそれぞれ異なるモダリティの特性を持って
おり、互いに補完的な関係にあることを示唆している。



\section{従来研究における課題と本研究の位置づけ}
自律移動ロボットの人追従走行において、オクルージョン発生時の堅牢性は未だ解決すべき
重要な課題である。 2D LiDARを用いた手法は計算コストが低い反面、取得情報が水平断面
に限られるため、オクルージョン後の再識別やロスト状態からの自律的な復帰が困難であ
る。 一方、3D LiDARを用いた手法は詳細な空間情報を取得できるが、既存研究の多くは追
跡の連続性に主眼を置いており、遮蔽により追跡が完全に途切れた後の再識別や復帰手法
については十分に確立されていない。 また、単眼カメラは外見情報を用いた再識別に優れ
るものの、視野の制約によるフレームアウトや照明条件の変化に弱いという課題がある。

そこで本研究では、ロボット向けの全方位3D LiDAR「Livox Mid-360」を用いた人追従シス
テムを提案する。全方位視野により対象のフレームアウトを防ぎつつ、点群データを活用
してオクルージョン後の再検出・再識別を行うことで、複雑な環境下でも自律的に追従を
継続できるシステムの構築を目指す。
%%%%%%%%%%%%%%%%%%%%%%%%%%%%%%%%%%%%%%%%%%%%%%%%%%%%%%%%%%%%%%%%%%%%%%%%%
\chapter{Method}
\section{概要}
本研究では、全方位3D LiDARを用いた人追従走行において、オクルージョン発生後のロスト
状態から自律的に復帰するため、LiDARベースの人物再同定モデル「ReID3D」を組み込んだ
システムを提案する。本システムは、主に「歩行者検出」「追従対象の登録・追跡」「オクル
ージョン後の再識別」の3つのフェーズから構成される。

まず、環境内の歩行者位置を特定するための検出モデルとして、3次元点群物体検出ネット
ワークである「PointPillars」を採用する。公開されているPointPillarsの学習済みモデル
は、回転式LiDAR(Velodyne HDL-64E)で収集されたKITTIデータセット
\cite{Are we ready for autonomous driving? the KITTI vision benchmark suite}
を用いているが、本研究で使用するプリズムスキャン方式のLivox Mid-360とはスキャンパ
ターンや点群密度が大きく異なる。このセンサ方式の違いに起因するドメインギャップは
検出精度の低下を招くため、本研究ではLivox Mid-360を用いて独自に構築した歩行者データ
セットを作成し、PointPillarsの再学習(ファインチューニング)を行うことで、
本システムに適した検出器を構築する。

システムの具体的な処理フローは次の通りである。まず、再学習されたPointPillarsを用い
て周囲の歩行者を検出する。追従開始時には、指定された対象の点群からReID3Dを用いて
特徴量を抽出し、追従対象として登録する。通常時は、検出位置に基づくトラッキングに
より対象への追従走行を行う。追従中に障害物による遮蔽(オクルージョン)が発生し
トラッキングが途切れた場合は、再識別フェーズへと移行する。このフェーズでは、
視野内で検出された全ての歩行者の点群をReID3Dに入力し、事前に登録された対象の特徴量
と比較照合を行う。これにより、オクルージョン後の環境から追従対象を正しく再識別し、
追従動作の自律的な再開を実現する。



\section{システム構成}
本研究で提案する人追従システムの全体構成を Fig. \ref{System overview.} に示す。
本システムは Robot Operating System 2 (ROS 2) をミドルウェアとして構築されており、
歩行者検出モデルには、PointPillars の ROS 2 実装である ros2\_tao\_pointpillars 
を採用した。 Fig. \ref{System overview.} は主に追従実行時における処理フローを示し
ているが、本システムの動作は、追従開始直後の「追従対象者の登録フェーズ」と、その後
の「追従フェーズ」で処理内容が大きく異なる。したがって、以下ではこれら2つのフェーズ
に分けて詳細を述べる。

\begin{figure*}[h]
    \begin{center}
    \includegraphics[height=100mm,clip]{figure/System_overview.png}
    \caption{System overview.}
    \label{System overview.}
    \end{center}
\end{figure*}

\paragraph{登録フェーズ}
本フェーズでは、まず3D LiDARから取得された点群データを ros2\_tao\_pointpillars へ
入力し、周囲の歩行者検出を行う。一般に、追従対象の初期化においてはロボットに最も近
接した検出結果を選定する手法がとられるが、検出器の誤認識(False Positive)により、
近傍の壁や障害物が誤って人物として検出される可能性がある。 このような非人物物体の
誤登録を防止するため、本システムでは探索範囲に空間的な制約を設ける。具体的には、
ロボット座標系において横幅 1.0 m、奥行き 3.0 m の矩形領域を設定し、この領域内に
存在する検出結果の中で、かつロボットに最も近い人物を追従対象として選定する。
対象が決定された後、システムは当該人物のトラッキングを開始すると同時に、ReID3Dを
用いて対象の点群から特徴量を抽出し、追従対象として登録する。

\paragraph{追従フェーズ}
本フェーズでは、3D LiDARから取得した点群データを ros2\_tao\_pointpillars へ入力し、
常時歩行者の検出を行う。検出された歩行者はトラッキングアルゴリズムによって追跡され、
前フレームまでの情報に基づいて追従対象との対応付けが行われる。システムはこのトラッ
キング結果に基づき、ロボットの移動制御指令を生成し、対象への追従走行を実行する。
一方、オクルージョン等により追従対象のトラッキングが 1.0 秒間途絶えた場合、システ
ムは対象を見失った(ロスト状態)と判定し、再識別処理へと移行する。 再識別処理では、
その時点で視野内に検出されている全ての歩行者の点群を ReID3D に入力して特徴量を抽出
する。得られた各歩行者の特徴量と、登録フェーズで事前に保存された追従対象の特徴量と
を比較照合することで、追従対象の再同定を行う。

\section{ソフトウェア構成}
ロボット用ノートPC内のソフトウェア構成をFig. \ref{Software stack.}に示す。
ロボット用ノートPCは、RTX 3070 8GB を搭載しており、オペレーティングシステムには
Ubuntu 22.04 LTSを使用した。深層学習モデルであるPointPillarsおよびReID3Dの実装に
あたっては,それぞれ独立したDockerコンテナ環境を構築している.これは,両モデルが
要求するCUDAツールキットやドライバのバージョンが異なり,単一のホスト環境内での共存
が困難であるという依存関係の問題を解決するためである. ミドルウェアにはROS2 Foxy
Fitzroyを採用しており,各DockerコンテナおよびホストPCはROS2の通信プロトコルを介
して連携している.また,PointPillarsによる検出からReID3Dによる識別,そしてロボット
の制御に至る一連のパイプラインを統合管理するため,独自のROS 2パッケージ「HARRP
 (Human-following Autonomous Robot system with ReID3D and PointPillars)」を開発
 した.本システムは,このHARRPパッケージを起動することで,分散されたコンテナ環境
 を意識することなく,統合システムとして実行可能である.

\begin{figure*}[h]
    \begin{center}
    \includegraphics[height=120mm,clip]{figure/Software_stack.png}
    \caption{Software stack.}
    \label{Software stack.}
    \end{center}
\end{figure*}

\section{データセットの作成}
\subsection{データ収集}
データ収集環境をFig.\ref{Data collection environment.}に示す。部屋の広さは
$6.0,\mathrm{m} \times 7.1,\mathrm{m}$である。LiDARは
Fig.\ref{LiDAR and custom stand.}のようにスタンドに固定し、部屋の中央に設置した。
本実験では、LiDARをロボットへ搭載することを想定し、
Fig.\ref{LiDAR and custom stand.}に示す専用スタンドを製作した。このスタンドは、
Fig.\ref{Height comparison.}に示すロボットのセンサ搭載位置と条件を一致させるよう
設計されており、LiDARの設置高さが地面から 119.5 mm となるように設定されている。


\begin{figure*}[h]
    \begin{center}
    \includegraphics[height=120mm,clip]{figure/data_collection_room.JPG}
    \caption{Data collection environment.}
    \label{Data collection environment.}
    \end{center}
\end{figure*}


\begin{figure*}[h]
  \begin{minipage}[b]{0.5\linewidth}
    \centering
    \includegraphics[keepaspectratio, width=7cm, angle=-90]{figure/lidar.JPG}
    \caption{LiDAR and custom stand.}
    \label{LiDAR and custom stand.}
  \end{minipage}
  \begin{minipage}[b]{0.5\linewidth}
    \centering
    \includegraphics[keepaspectratio, height=7cm]{figure/kachaka_lidar.JPG}
    \caption{Height comparison.}
    \label{Height comparison.}
  \end{minipage}
\end{figure*}

\clearpage

%===========================================================
% 1ページ目:前半 18枚
%===========================================================
\begin{figure}[p]
    \centering
    % リスト内の改行コードがスペース化しないよう % を付加し、カンマ後のスペースも削除
    \foreach \n [count=\i] in {%
        9496,9497,9498,9499,9500,9502,%
        9503,9504,9505,9506,9507,9508,%
        9509,9510,9511,9512,9513,9514%
    }{
        \subfigure[No.\n]{
            % ↓ ここも念のため確認: \n.JPG の間にスペースがないか
            \includegraphics[width=0.3\linewidth]{figure/IMG_\n.JPG}
        }%
        \ifnum\numexpr\i-((\i-1)/3)*3\relax=0 \par\vspace{1mm} \else \hfill \fi
    }
    \caption{実験画像一覧 (1/2)}
\end{figure}

\subsection{アノテーション}

\subsection{データ拡張}



\section{PointPillarsの再学習}



\section{トラッキング}



\section{ロボット制御}
%%%%%%%%%%%%%%%%%%%%%%%%%%%%%%%%%%%%%%%%%%%%%%%%%%%%%%%%%%%%%%%%%%%%%%%%%
\chapter{Experiments}
\section{実験目的}
本研究では、3D-LiDARを用いた人追従において、障害物による遮蔽(オクルージョン)が発生した
後も、対象を再識別し追跡を継続できるシステムの開発を目的としている。 本システムの有効性を
検証するため、実験ではセンサとしてLivox社製のロボット向け3D-LiDARを採用した。本センサは
独自の走査パターンを有するため、その点群特性に最適化した物体検出ネットワーク
PointPillarsを用い、歩行者検出精度を定量的に評価する。また、公開データセットを用い、
開発したシステムがオクルージョン後の対象人物を正しく再識別できるかについても性能評価を行う。
以上の実験を通じ、3D-LiDARを用いた歩行者検出から再識別に至る一連の性能を明らかにし、構築した
人追従システムの有用性を実証する。

\section{実験方法}
提案システムの有効性を検証するために2種類の実験を行う。第一に、Livox社製3D-LiDARの点群
特性に最適化を施したPointPillarsの歩行者検出精度を定量的に評価する実験である。第二に、構築
した人追従システム全体の追跡性能および再識別性能を評価する実験である。
また、実験機材にはRTX 3070 8GBを搭載したノートPCを使用した。

\subsection{歩行者検出性能に関する比較実験}
本実験では、3.4節で構築した4,500フレームの検証用データセットを用い、PointPillarsの歩行者検出
性能を評価する。評価指標には、物体検出において広く用いられる BEV (Bird’s Eye View) AP 
および 3D AP を採用した。 BEV APは、推定された3次元バウンディングボックスを2次元平面
(鳥瞰図)に投影し、グランドトゥルースとの領域の重複率に基づいて算出される指標である。
一方、3D APは、バウンディングボックスの3次元的な位置および体積の重複率 
(IoU: Intersection over Union) に基づいて算出される。なお、本実験におけるIoUの閾値は、
歩行者検出の一般的な基準に従い 0.5 に設定した。
推論実験の環境構築にあたり、入力データは点群のバイナリファイルをROS 2トピックとして変換
・パブリッシュし、これを検出モデルがサブスクライブする形式を採用した。この際、トピックの
パブリッシュ周波数は 10 Hz に設定している。 また、本手法の有効性を検証するための比較対象
として、既存のオープンソース実装である ros2\_tao\_pointpillars パッケージの標準
モデルを用いた。

\subsection{追跡・再識別性能に関する比較実験}
本実験では、公開データセットを用い、構築した人追従システム全体の追跡性能および再識別性能
を評価する。 評価用データセットには、TPT-BENCH
\cite{TPT-Bench: A Large-Scale, Long-Term and Robot-Egocentric Dataset for 
Benchmarking Target Person Tracking} を採用した。ロボット知覚に関する既存のデータセット
としては、RoboSense\cite{RobSense: A Robust Multi-modal Foundation Model for 
Remote Sensing with Static, Temporal, and Incomplete Data Adaptability} や JRDB
\cite{JRDB-PanoTrack: An Open-world Panoptic Segmentation and Tracking Robotic 
Dataset in Crowded Human Environments} などが挙げられる。しかし、本研究の主眼である
「特定の人物を継続的に追跡するタスク (Target Person Tracking: TPT)」における評価には、
TPTに特化したデータセットが最適であるためTPT-BENCHを選定した。 TPT-BENCHは、LiDAR点群、
RGB画像、深度情報の3種類のセンサ情報を含んでおり、屋内・屋外、あるいは日中・夜間といった
多様な環境下で収集された大規模データセットである。

評価指標には、Average Overlap (AO)、F1-score、および Avg Max Recall (AMR) の3つを用
いる。 AOは、推定されたバウンディングボックスとグランドトゥルース(正解データ)との領域
重複度(IoU)の平均値を示す指標である。F1-scoreは、追跡の適合率 (Precision) と再現率 
(Recall) の調和平均であり、検出の正確さと網羅性のバランスを評価する。AMRは、ターゲットを
どの程度の割合で正しく捉え続けられたかを示す、追跡の頑健性を測る指標である。



\section{実験結果}
\subsection{歩行者検出性能に関する比較実験}
歩行者検出の実験結果をTable \ref{Evaluation results of pedestrian detection.}に示す。
ros2\_tao\_pointpillarsのBEV APは35.47[\%]、3D APは18.50[\%]であったのに対し、
本手法はBEV APが79.86[\%]、3D APが61.59[\%]であり、いずれの指標においても従来のモデル
を大幅に上回る結果となった。どちらのモデルも誤検出があり、ros2\_tao\_pointpillarsでは、
2人の歩行者を1つのバウンディングボックスとして検出するケースが見られた。一方、本手法では、
概ね歩行者を検出できていたが、歩行者が停止し、直立している場合に検出されない現象が見られた。

\begin{table}[h]
  \centering
  \caption{Evaluation results of pedestrian detection.}
  \label{Evaluation results of pedestrian detection.}
  \begin{tabular}{ccc} \toprule
    Model & BEV AP (\%) & 3D AP (\%) \\ \midrule
    ros2\_tao\_pointpillars & 35.47 & 18.50 \\
    Ours & 79.86 & 61.59 \\ \bottomrule
  \end{tabular}
\end{table}


\section{考察}
実験結果から、Livox社製3D-LiDARの点群特性に最適化を施したPointPillarsが、
従来のモデルを大幅に上回る歩行者検出性能を示したことが確認できた。これは、
本研究で提案した独自のデータセットを用いた学習が、センサ特性に適応した特徴抽出を可能にし、
歩行者検出においてドメインギャップを効果的に克服したためであると考えられる。
ただし、誤検出の傾向には両モデルで違いが見られた。ros2\_tao\_pointpillarsでは、
近接する歩行者を1つのバウンディングボックスとして誤検出するケースが見られた。
これは、公開されている学習済みモデルでは、2人の歩行者を1つのバウンディングボックスで
表現するようなアノテーションが存在したことが影響している可能性がある。
一方、本手法では、歩行者が停止し直立している場合に検出されない現象が見られた。これは、
データセットにおいて、歩行者が動いている状態の点群が多く含まれていたことが影響している
と考えられる。
%%%%%%%%%%%%%%%%%%%%%%%%%%%%%%%%%%%%%%%%%%%%%%%%%%%%%%%%%%%%%%%%%%%%%%%%%
\chapter{Conclusion}
\section{まとめ}
本研究では、3D-LiDARを用いた人追従システムの開発を行い、歩行者検出性能および追跡・再識別
性能の検証を行った。 本システムでは、Livox社製3D-LiDARの点群特性に最適化を施した
PointPillarsを採用し、独自に構築したデータセットで学習を行うことで高精度な歩行者検出器
を構築した。さらに、ReID3Dを統合することで、オクルージョン(遮蔽)発生後も対象人物を再識別
可能な人追従システムを実現した。 提案手法の有効性を検証するため、独自データセットおよび
公開データセットを用いた評価実験を行った結果、最適化を施したPointPillarsが従来モデルを
大幅に上回る歩行者検出性能を示し、本手法の有用性が確認された。

\section{今後の課題}
今後の課題として、以下の二点が挙げられる。 第一に、歩行者検出性能のさらなる向上である。
本実験の結果より、歩行者が停止し直立している場合に検出漏れが発生する傾向が確認された。
この原因として、学習に使用したデータセットに歩行者が動作している状態の点群が支配的であり、
静止状態のデータが不足していたことが考えられる。したがって、歩行者が停止している状態の
点群を拡充したデータセットを新たに構築し、再学習を行うことで、検出性能の改善が見込まれる。
第二に、追跡および再識別性能の向上である。本システムではトラッキングに線形カルマンフィルタ
を採用しているが、歩行者やロボットの急激な動作により相対軌跡が非線形となる場面では、
追従が困難になる課題が残された。この解決策として、拡張カルマンフィルタ (EKF) や
アンセンテッドカルマンフィルタ (UKF) といった非線形推定手法の導入が有効であると考えられる。
また、再識別モデルとして用いたReID3Dは、交差点監視などを想定しており、対象との距離が遠く
全身が点群に含まれることを前提としている。しかし、人追従タスクではLiDARと歩行者の距離が
近く、画角外れなどにより身体の一部のみしか計測されない状況が頻発する。そのため、近距離
かつ部分的な点群情報からでもロバストに再識別可能な新たなモデルの開発が必要である。
%%%%%%%%%%%%%%%%%%%%%%%%%%%%%%%%%%%%%%%%%%%%%%%%%%%%%%%%%%%%%%%%%%%%%%%%%
\chapter*{Acknowledgments}
\indent
本プロジェクトを行うにあたり全体を通してご指導、ご教授、議論などのご助力をいただきました本学ロボティクス学科の出村公成教授に深く感謝いたします。
また、データセットの作成や実験にご助力いただいた、出村研究室の皆様にお礼申し上げます。
最後に、これまで学生生活を支えていただいた両親に深く感謝いたします。

\begin{flushright}
令和6年2月8日
\end{flushright}

%%%%%%%%%%%%%%%%%%%%%%%%%%%%%%%%%%%%%%%%%%%%%%%%%%%%%%%%%%%%%%%%%%%%%%%%%
\begin{thebibliography}{99}
    \bibitem{統計からみた我が国の高齢者}
    総務省統計局, "統計からみた我が国の高齢者", (https://www.stat.go.jp/data/topics/pdf/topics138.pdf, 2024年2月1日閲覧).

    \bibitem{高齢化の現状と将来像}
    内閣府, "高齢化の現状と将来像", (https://www8.cao.go.jp/kourei/whitepaper/w-2023/zenbun/pdf/1s1s\_01.pdf, 2024年2月1日閲覧).
    
    \bibitem{近年の建設工事用ロボット開発について}
    宮口幹太, “近年の建設工事用ロボット開発について” 計測と制御 第61巻 第9号, p. 641-644, 2022.
    
    \bibitem{既存AGVを超える特長を持った協働運搬ロボット「サウザー」}
    大島章, 城吉宏泰, 柄川索, 松下裕介, 阪東茂, "既存AGVを超える特長を持った協働運搬ロボット「サウザー」", 日本ロボット学会誌 第39巻 第1号, p. 65-66, 2021.

    \bibitem{深層学習を用いた人追従機能の開発}
    飯田一成, 出村公成, "深層学習を用いた人追従機能の開発", 
    令和4年度金沢工業大学工学部ロボティクス学科プロジェクトデザインI\hspace{-1.2pt}I\hspace{-1.2pt}I,
    2023.

    \bibitem{People Detection and Tracking Using LIDAR Sensors}
    Claudia Álvarez-Aparicio, Ángel Manuel Guerrero-Higueras, 
    Francisco Javier Rodríguez-Lera, Jonatan Ginés Clavero, 
    Francisco Martín Rico and Vicente Matellán, 
    "People Detection and Tracking Using LIDAR Sensors",
    Robotics 2019, 8, 75.

    \bibitem{Temporal convolutional networks for multi-person activity recognition using a 2D LIDAR}
    Fei Luo, Stefan Poslad, and Eliane Bodanese,
    "Temporal convolutional networks for multi-person activity recognition using a 2D LIDAR",
    IEEE Internet of Things Journal, Volume: 7, Issue: 8, 2020.

    \bibitem{Tracking People in a Mobile Robot From 2D LIDAR Scans Using Full Convolutional Neural Networks for Security in Cluttered Environments}
    Ángel Manuel Guerrero-Higueras, Claudia Álvarez-Aparicio, 
    María Carmen Calvo Olivera, Francisco J. Rodríguez-Lera, 
    Camino Fernández-Llamas, Francisco Martín Rico and Vicente Matellán,
    "Tracking People in a Mobile Robot From 2D LIDAR Scans Using Full Convolutional Neural Networks for Security in Cluttered Environments",
    Frontiers in Neurorobotics, Volume 12, Article 85, 2019.

    \bibitem{Tracking People Using Ankle-Level 2D LiDAR for Gait Analysis}
    Mahmudul Hasan, Junichi Hanawa, Riku Goto, Hisato Fukuda, 
    Yoshinori Kuno and Yoshinori Kobayashi,
    "Tracking People Using Ankle-Level 2D LiDAR for Gait Analysis",
    Advances in Artificial Intelligence, Software and Systems Engineering,
    pp 40-46, 2020

    \bibitem{A Robust Autonomous Following Method for Mobile Robots in Dynamic Environments}
    DAPING JIN , ZHENG FANG, (Member, IEEE), AND JIEXIN ZENG, 
    "A Robust Autonomous Following Method for Mobile Robots in Dynamic Environments", 
    IEEE Access, Volume: 8, pp. 150311-150325, 2020.

\end{thebibliography}

%%%%%%%%%%%%%%%%%%%%%%%%%%%%%%%%%%%%%%%%%%%%%%%%%%%%%%%%%%%%%%%%%%%%%%%%%

%%%%%%%%%%%%%%%%%%%%%%%%%%%%%%%%%%%%%%%%%%%%%%%%%%%%%%%%%%%%%%%%%%%%%%%%%
\chapter*{Appendices}
%%%%%%%%%%%%%%%%%%%%%%%%%%%%%%%%%%%%%%%%%%%%%%%%%%%%%%%%%%%%%%%%%%%%%%%%%
%ここからソースコードの表示に関する設定
\lstset{
  basicstyle={\ttfamily},
  identifierstyle={\small},
  commentstyle={\smallitshape},
  keywordstyle={\small\bfseries},
  ndkeywordstyle={\small},
  stringstyle={\small\ttfamily},
  frame={tb},
  breaklines=true,
  columns=[l]{fullflexible},
  numbers=left,
  xrightmargin=0zw,
  xleftmargin=3zw,
  numberstyle={\scriptsize},
  stepnumber=1,
  numbersep=1zw,
  lineskip=-0.5ex
}
%ここまでソースコードの表示に関する設定
パラメータの設定ファイルを、ソースコード\ref{param}に示す。
\begin{lstlisting}[caption=follow\_me\_params.yaml, label=param]
/follow_me/laser_to_img:
  ros__parameters:
    # 縮小サイズを取得. 1[px] = 0.01[m]
    discrete_size: 0.01
    # Max LiDAR Range
    max_lidar_range: 3.5
    # 画像を表示するフラッグ
    img_show_flg: False

/follow_me/person_detector:
  ros__parameters:
    # 追従対象者との距離
    target_dist: 0.5
    # 追従対象を見失ったときに追従を再開する時の距離の誤差
    target_diff: 0.3
    # 追従ポイント(制御を止める領域)の半径
    target_radius: 0.1
    # 人を検出する範囲(円)の半径
    target_range: 0.4
    # 起動時に追従対象者を検出するまでの待機時間
    init_time: 3.0
    # 起動時に追従対象を検出するまでのflg
    none_person_flg: True
  
/follow_me/base_controller:
  ros__parameters:
    # ロボットからみてtolerance[°]以内だったら積分制御しない視野角
    tolerance: 1.0
    # 積分制御をし始める視野角[°]
    i_range: 3.0
    # 並進のPゲイン==========================
    lkp: 0.3
    # 旋回のPIDゲイン========================
    # Pゲイン
    akp: 0.005
    # Iゲイン
    aki: 0.0002
    # Dゲイン
    akd: 0.0009
\end{lstlisting}

2D-LiDARの距離データから俯瞰画像を生成するソースコードを、ソースコード\ref{image}に示す。
\begin{lstlisting}[caption=laser\_to\_image.py, label=image]
import numpy as np
import os
import sys
import cv2
import math
import rclpy
from rclpy.node import Node
from rclpy.parameter import Parameter
from sensor_msgs.msg import LaserScan, Image
from rclpy.qos import qos_profile_sensor_data
from rcl_interfaces.msg import SetParametersResult
from cv_bridge import CvBridge, CvBridgeError
# Custom
from .modules.gradient import gradation_3d_img as gradation


class LaserToImg(Node):
    def __init__(self):
        super().__init__('laser_to_img')
        # Publisher
        self.pub = self.create_publisher(Image, '/follow_me/laser_img', 10)
        # Subscriber
        self.create_subscription(LaserScan, '/scan', self.cloud_to_img_callback, qos_profile_sensor_data)
        # OpenCV
        self.bridge = CvBridge()
        # Parameters
        self.declare_parameters(
                namespace='',
                parameters=[
                    ('discrete_size', Parameter.Type.DOUBLE),
                    ('max_lidar_range', Parameter.Type.DOUBLE),
                    ('img_show_flg', Parameter.Type.BOOL)])
        self.add_on_set_parameters_callback(self.param_callback)
        # Get parameters
        self.param_dict = {}
        self.param_dict['discrete_size'] = self.get_parameter('discrete_size').value
        self.param_dict['max_lidar_range'] = self.get_parameter('max_lidar_range').value
        self.param_dict['img_show_flg'] = self.get_parameter('img_show_flg').value
        # Values
        self.color_list = gradation([0,0,255], [255,0,0], [1, 100], [True,True,True])[0]
        # Output
        self.output_screen()

    def output_screen(self):
        for key, value in self.param_dict.items():
            self.get_logger().info(f"{key}: {value}")

    def param_callback(self, params):
        for param in params:
            self.param_dict[param.name] = param.value
            self.get_logger().info(f"Set param: {param.name} >>> {param.value}")
        return SetParametersResult(successful=True)

    def cloud_to_img_callback(self, scan):
        
        # discrete_factor
        discrete_factor = 1/self.param_dict['discrete_size']
        # max_lidar_rangeとdiscrete_factorを使って画像サイズを設定する
        img_size = int(self.param_dict['max_lidar_range']*2*discrete_factor)

        # LiDARデータ
        maxAngle = scan.angle_max
        minAngle = scan.angle_min
        angleInc = scan.angle_increment
        maxLength = scan.range_max
        ranges = scan.ranges
        intensities = scan.intensities
        #intensities = scan.intensities
        
        # 距離データの個数を格納
        num_pts = len(ranges)
        # 721行2列の空行列を作成
        xy_scan = np.zeros((num_pts, 2))
        # 3チャンネルの白色ブランク画像を作成
        blank_img = np.zeros((img_size, img_size, 3), dtype=np.uint8) + 255
        # rangesの距離・角度からすべての点をXYに変換する処理
        for i in range(num_pts):
            # 範囲内かを判定
            if (ranges[i] > self.param_dict['max_lidar_range']) or (math.isnan(ranges[i])):
                pass
            else:
                # 角度とXY座標の算出処理
                angle = minAngle + float(i)*angleInc
                xy_scan[i][1] = float(ranges[i]*math.cos(angle))  # y座標
                xy_scan[i][0] = float(ranges[i]*math.sin(angle))  # x座標

        # ブランク画像にプロットする処理
        for i in range(num_pts):
            pt_x = xy_scan[i, 0]
            pt_y = xy_scan[i, 1]
            if (pt_x < self.param_dict['max_lidar_range']) or (pt_x > -1*(self.param_dict['max_lidar_range']-self.param_dict['discrete_size'])) or (pt_y < self.param_dict['max_lidar_range']) or (pt_y > -1 * (self.param_dict['max_lidar_range']-self.param_dict['discrete_size'])):
                pix_x = int(math.floor((pt_x + self.param_dict['max_lidar_range']) * discrete_factor))
                pix_y = int(math.floor((self.param_dict['max_lidar_range'] - pt_y) * discrete_factor))
                if (pix_x > img_size) or (pix_y > img_size):
                    print("Error")
                else:
                    blank_img[pix_y, pix_x] = [0, 0, 0]

        # CV2画像からROSメッセージに変換してトピックとして配布する
        img = self.bridge.cv2_to_imgmsg(blank_img, encoding="bgr8")
        self.pub.publish(img)

        # 画像の表示処理. imgshow_flgがTrueの場合のみ表示する
        if self.param_dict['img_show_flg']:
            cv2.imshow('laser_img', blank_img)
            cv2.waitKey(3)
            #更新のため一旦消す
            blank_img = np.zeros((img_size, img_size, 3))
        else:
            pass

def main():
    rclpy.init()
    node = LaserToImg()
    try:
        rclpy.spin(node)
    except KeyboardInterrupt:
        pass
    node.destroy_node()
    rclpy.shutdown()
\end{lstlisting}

追従目標を特定するソースコードを、ソースコード\ref{person}に示す。
\begin{lstlisting}[caption=person\_detector.py, label=person]
import math
import time
import cv2
import rclpy
from rclpy.node import Node
from rclpy.parameter import Parameter
from rcl_interfaces.msg import SetParametersResult, ParameterEvent
from rcl_interfaces.srv import GetParameters
from sensor_msgs.msg import Image
from geometry_msgs.msg import Point
from cv_bridge import CvBridge, CvBridgeError
from yolov8_msgs.msg import DetectionArray


class PersonDetector(Node):
    def __init__(self):
        super().__init__('person_detector')
        # OpenCV Bridge
        self.bridge = CvBridge()
        # Publisher
        self.point_pub = self.create_publisher(Point, '/follow_me/target_point', 10)
        self.img_pub = self.create_publisher(Image, '/follow_me/image', 10)
        # Subscriber
        self.create_subscription(DetectionArray, '/yolo/detections', self.yolo_callback, 10)
        self.create_subscription(Image, '/yolo/dbg_image', self.img_show, 10)
        self.create_subscription(ParameterEvent, '/parameter_events', self.param_event_callback, 10)
        # Service
        self.srv_client = self.create_client(GetParameters, '/follow_me/laser_to_img/get_parameters')
        while not self.srv_client.wait_for_service(timeout_sec=0.5):
            self.get_logger().info('/follow_me/laser_to_img server is not available ...')
        # Parameters
        self.declare_parameters(
                namespace='',
                parameters=[
                    ('target_dist', Parameter.Type.DOUBLE),
                    ('init_time', Parameter.Type.DOUBLE),
                    ('none_person_flg', Parameter.Type.BOOL),
                    ('target_diff', Parameter.Type.DOUBLE),
                    ('target_radius', Parameter.Type.DOUBLE),
                    ('target_range', Parameter.Type.DOUBLE)])
        self.add_on_set_parameters_callback(self.param_callback)
        # Get parameters
        self.param_dict ={}
        self.param_dict['target_dist'] = self.get_parameter('target_dist').value
        self.param_dict['init_time'] = self.get_parameter('init_time').value
        self.param_dict['none_person_flg'] = self.get_parameter('none_person_flg').value
        self.param_dict['target_diff'] = self.get_parameter('target_diff').value
        self.param_dict['target_radius'] = self.get_parameter('target_radius').value
        self.param_dict['target_range'] = self.get_parameter('target_range').value
        self.param_dict['discrete_size'] = self.get_param()  # laser_to_imgからもってくる
        # Value
        self.person_list = []
        self.target_data = []  # 追従対象のデータを保存するリスト
        self.before_data = [0.0, 0.0, 0.0]
        self.center_x = 0.0
        self.center_y = 0.0
        self.target_point = Point()
        self.target_px = []
        self.laser_img = 0.0
        self.height = 0.0
        self.width = 0.0
        # Output
        self.output_screen()

    def output_screen(self):
        for key, value in self.param_dict.items():
            self.get_logger().info(f"{key}: {value}")

    def param_event_callback(self, receive_msg):
        for data in receive_msg.changed_parameters:
            if data.name == 'discrete_size':
                self.param_dict['discrete_size'] = data.value.double_value
                self.get_logger().info(f"Param event: {data.name} >>> {self.param_dict['discrete_size']}")

    def param_callback(self, params):
        for param in params:
            self.param_dict[param.name] = param.value
            self.get_logger().info(f"Set param: {param.name} >>> {param.value}")
        return SetParametersResult(successful=True)
    
    def get_param(self):
        req = GetParameters.Request()
        req.names = ['discrete_size']
        future = self.srv_client.call_async(req)
        while rclpy.ok():
            rclpy.spin_once(self, timeout_sec=0.1)
            if future.done():
                break
        return future.result().values[0].double_value

    def yolo_callback(self, receive_msg):
        if not receive_msg.detections:
            self.center_x = self.center_y = None
        else:
            self.person_list.clear()
            for person in receive_msg.detections:
                px = Point()
                px.x = person.bbox.center.position.x
                px.y = person.bbox.center.position.y
                self.person_list.append(px)

    def plot_robot_point(self):
        # 画像の中心を算出
        robot_x = round(self.width / 2)
        robot_y = round(self.height / 2)
        # 描画処理
        cv2.circle(img = self.laser_img,
                   center = (round(robot_x), round(robot_y)),
                   radius = 5,
                   color = (0, 255, 0),
                   thickness = -1)

    def plot_person_point(self):
        cv2.circle(img = self.laser_img,
                   center = (round(self.center_x), round(self.center_y)),
                   radius = 8,
                   color = (0, 0, 255),
                   thickness = -1)

    def plot_target_point(self):
        cv2.circle(img = self.laser_img,
                   center = (int(self.target_px[0]), int(self.target_px[1])),
                   radius = int(self.param_dict['target_radius']/self.param_dict['discrete_size']),
                   color = (255, 0, 0),
                   thickness = 2)

    def plot_target_range(self):
        cv2.circle(img = self.laser_img,
                   center = (int(self.before_data[2][0]), int(self.before_data[2][1])),
                   radius = int(self.param_dict['target_range']/self.param_dict['discrete_size']),
                   color = (196, 0, 255),
                   thickness = 2)

    def diff_distance(self, data):
        return abs(data - self.before_data[0])

    def euclidean_distance(self, data, before_data):
        return math.sqrt((data.x-before_data.x)**2 + (data.y-before_data.y)**2)

    def select_target(self, robot_px_x, robot_px_y):
        # personまでの距離と座標のリストを作成
        self.target_data.clear()
        for person_px in self.person_list:
            person_point = Point()
            person_point.x = (robot_px_x - person_px.y)*self.param_dict['discrete_size']
            person_point.y = (robot_px_y - person_px.x)*self.param_dict['discrete_size']
            distance = math.sqrt(person_point.x**2 + person_point.y**2)
            self.target_data.append([distance, person_point, [person_px.x, person_px.y]])
        # 0番目に1時刻前の追従対象との距離の誤差を格納する
        self.target_data = [[self.diff_distance(data[0]), data[1], data[2]] for data in self.target_data]
        # 検出範囲内のpersonを追従対象とする(起動時だけ一番近い人を追従対象にする)
        if self.param_dict['none_person_flg']:
            target = min(self.target_data)
            param_bool = Parameter('none_person_flg', Parameter.Type.BOOL, False)
            self.set_parameters([param_bool])
        else:
            for data in self.target_data:
                diff = self.euclidean_distance(data[1], self.before_data[1])
                if diff <= self.param_dict['target_range']:
                    target = data
                    break
                else:
                    target = None
        # targetがNoneだったらself.before_dataを初期化してnone_person_flgをTrueにする
        if target is None:
            #self.before_data = [0.0, 0.0, 0.0]
            #param_bool = Parameter('none_person_flg', Parameter.Type.BOOL, True)
            #self.set_parameters([param_bool])
            pass
        else:
            # 計算のために距離を保存
            self.before_data = target
        return target

    def generate_target(self):
        self.target_px.clear()
        # 画像の中心を算出
        robot_x = self.height / 2
        robot_y = self.width / 2
        # 追従目標を選定
        target_person = self.select_target(robot_x, robot_y)
        if not target_person is None:
            result_point = target_person[1]
            self.center_x = target_person[2][0]
            self.center_y = target_person[2][1]
        else:
            result_point = False
        return result_point


    def img_show(self, receive_msg):
        self.laser_img = self.bridge.imgmsg_to_cv2(receive_msg, desired_encoding='bgr8')
        self.height, self.width, _ = self.laser_img.shape[:3]
        # ロボットの座標をプロット
        self.plot_robot_point()
        # 追従対象の検出範囲をプロット
        if not self.param_dict['none_person_flg']:
            self.plot_target_range()
        # personがいるか判定
        if self.person_list:
            # 追従対象を生成
            target_point = self.generate_target()
            # 追従対象がいなければロボット台車を停止する
            if not target_point:
                self.target_point.x = 0.0
                self.target_point.y = 0.0
                self.point_pub.publish(self.target_point)
            else:
                robot_x = self.height / 2
                robot_y = self.width / 2
                # 目標座標を生成(px): 横x, 縦y
                target_x = self.center_x
                target_y = self.center_y + (self.param_dict['target_dist']/self.param_dict['discrete_size'])
                self.target_px.append(target_x)
                self.target_px.append(target_y)
                # 目標座標を生成(m): 縦x, 横y(ロボット座標系に合わせる)
                self.target_point.x = (robot_x - target_y)*self.param_dict['discrete_size']
                self.target_point.y = (robot_y - target_x)*self.param_dict['discrete_size']
                # パブリッシュ
                self.point_pub.publish(self.target_point)
                # グラフに描画
                self.plot_target_point()
                self.plot_person_point()

        # ros2 bag 用にトピックとして画像を配布
        img = self.bridge.cv2_to_imgmsg(self.laser_img, encoding="bgr8")
        self.img_pub.publish(img)

        # 画像を表示
        cv2.imshow('follow_me', self.laser_img)
        cv2.waitKey(1)


def main():
    rclpy.init()
    node = PersonDetector()
    try:
        rclpy.spin(node)
    except KeyboardInterrupt:
        pass
    node.destroy_node()
    rclpy.shutdown()
\end{lstlisting}

ロボット台車を制御するソースコードを、ソースコード\ref{base}に示す。
\begin{lstlisting}[caption=base\_controller.py, label=base]
import time
import math
import rclpy
from rclpy.node import Node
from rclpy.parameter import Parameter
from rcl_interfaces.msg import SetParametersResult, ParameterEvent
from rcl_interfaces.srv import GetParameters
from nav_msgs.msg import Odometry
from geometry_msgs.msg import Twist, Point


class BaseController(Node):
    def __init__(self):
        super().__init__('base_controller')
        # Publisher
        self.pub = self.create_publisher(Twist, '/cmd_vel', 10)
        self.data_pub = self.create_publisher(Point, '/follow_me/distance_angle_data', 10)
        # Subscriber
        self.create_subscription(Point, '/follow_me/target_point', self.callback, 10)
        self.create_subscription(Odometry, '/odom', self.odom_callback, 10)
        self.create_subscription(ParameterEvent, '/parameter_events', self.param_event_callback, 10)
        # Service
        self.srv_client = self.create_client(GetParameters, '/follow_me/person_detector/get_parameters')
        while not self.srv_client.wait_for_service(timeout_sec=0.5):
            self.get_logger().info('/follow_me/laser_to_img server is not available ...')
        # Parameters
        self.declare_parameters(
                namespace='',
                parameters=[
                    ('tolerance', Parameter.Type.DOUBLE),
                    ('i_range', Parameter.Type.DOUBLE),
                    ('lkp', Parameter.Type.DOUBLE),
                    ('akp', Parameter.Type.DOUBLE),
                    ('aki', Parameter.Type.DOUBLE),
                    ('akd', Parameter.Type.DOUBLE)])
        self.add_on_set_parameters_callback(self.param_callback)
        # Get parameters
        self.param_dict = {}
        self.param_dict['tolerance'] = self.get_parameter('tolerance').value
        self.param_dict['i_range'] = self.get_parameter('i_range').value
        self.param_dict['lkp'] = self.get_parameter('lkp').value
        self.param_dict['akp'] = self.get_parameter('akp').value
        self.param_dict['aki'] = self.get_parameter('aki').value
        self.param_dict['akd'] = self.get_parameter('akd').value
        self.param_dict['target_radius'] = self.get_param()  # person_detectorからもってくる
        # Value
        self.twist = Twist()
        self.target_angle = 0.0
        self.target_distance = 0.0
        self.target_x = 0.0
        self.target_y = 0.0
        self.delta_t = 0.0
        self.robot_angular_vel = 0.0
        # Output
        self.output_screen()

    def output_screen(self):
        for key, value in self.param_dict.items():
            self.get_logger().info(f"{key}: {value}")

    def param_event_callback(self, receive_msg):
        for data in receive_msg.changed_parameters:
            if data.name == 'target_radius':
                self.param_dict['target_radius'] = data.value.double_value
                self.get_logger().info(f"Param event: {data.name} >>> {self.param_dict['target_radius']}")

    def param_callback(self, params):
        for param in params:
            self.param_dict[param.name] = param.value
            self.get_logger().info(f"Set param: {param.name} >>> {param.value}")
        return SetParametersResult(successful=True)

    def get_param(self):
        req = GetParameters.Request()
        req.names = ['target_radius']
        future = self.srv_client.call_async(req)
        while rclpy.ok():
            rclpy.spin_once(self, timeout_sec=0.1)
            if future.done():
                break
        return future.result().values[0].double_value

    def point_to_angle(self, point):
        return math.degrees(math.atan2(point.y, point.x))

    def point_to_distance(self, point):
        distance = math.sqrt(point.x**2 + point.y**2)
        if point.x < 0.0:
            distance = -distance
        return distance

    def callback(self, receive_msg):
        self.target_x = receive_msg.x
        self.target_y = receive_msg.y
        self.target_distance = self.point_to_distance(receive_msg)
        self.target_angle = self.point_to_angle(receive_msg)

    def odom_callback(self, receive_msg):
        self.robot_angular_vel = receive_msg.twist.twist.angular.z

    # 比例制御量計算
    def p_control(self):
        return self.param_dict['akp']*self.target_angle

    # 微分制御量計算
    def d_control(self, p_term):
        return self.param_dict['akd']*(p_term - self.robot_angular_vel)

    # 積分制御量計算
    def i_control(self, p_term, d_term):
        value = 0.0
        diff = (p_term + d_term) - self.robot_angular_vel

        if not diff < self.param_dict['tolerance'] and diff < self.param_dict['i_range']:
            value = self.param_dict['aki']*self.target_angle*self.delta_t

        return value

    def pid_update(self):
        # 制御量を計算
        p_term = self.p_control()
        d_term = self.d_control(p_term)
        i_term = self.i_control(p_term, d_term)

        linear_vel = self.param_dict['lkp']*self.target_distance
        angular_vel = -1*(p_term + i_term + d_term)

        if linear_vel < 0.0:
            linear_vel = 0.0
            angular_vel = 0.0

        return linear_vel, angular_vel

    def in_range(self):
        result = False
        if abs(self.target_distance) <= self.param_dict['target_radius']:
            result = True
        return result

    def execute(self, rate=100):
        start_time = time.time()
        before_time = 0.0

        while rclpy.ok():
            self.delta_t = time.time() - start_time
            rclpy.spin_once(self)

            # 許容範囲内外を判
            if self.in_range():
                linear_vel = 0.0
                angular_vel = 0.0
            else:
                linear_vel, angular_vel = self.pid_update()

            # 制御量をパブリッシュ
            self.twist.linear.x = linear_vel
            self.twist.angular.z = angular_vel
            self.pub.publish(self.twist)

            # 実験用に目標までの距離と角度をパブリッシュ
            data = Point()
            data.x = self.target_distance
            data.z = self.target_angle
            self.data_pub.publish(data)

            time.sleep(1/rate)


def main():
    rclpy.init()
    node = BaseController()
    try:
        node.execute()
    except KeyboardInterrupt:
        pass

    node.destroy_node()
    rclpy.shutdown()
\end{lstlisting}

以上のソースコードをまとめて起動するソースコードを、ソースコード\ref{launch}に示す。
\begin{lstlisting}[caption=follow\_me.launch.py, label=launch]
import os
from ament_index_python.packages import get_package_share_directory
import launch
from launch import LaunchDescription
from launch.actions import DeclareLaunchArgument
from launch_ros.actions import Node

def generate_launch_description():
    config = os.path.join(
        get_package_share_directory('recognition_by_lidar'),
        'config',
        'follow_me_params.yaml')

    namespace = 'follow_me'

    return LaunchDescription([
        Node(
            namespace=namespace,
            package='recognition_by_lidar',
            executable='laser_to_img',
            name='laser_to_img',
            parameters=[config],
            output='screen',
            respawn=True),
        Node(
            namespace=namespace,
            package='recognition_by_lidar',
            executable='person_detector',
            name='person_detector',
            parameters=[config],
            output='screen',
            respawn=True),
        Node(
            namespace=namespace,
            package='recognition_by_lidar',
            executable='base_controller',
            name='base_controller',
            parameters=[config],
            output='screen',
            respawn=True,
            on_exit=launch.actions.Shutdown()),
        ])
\end{lstlisting}
%%%%%%%%%%%%%%%%%%%%%%%%%%%%%%%%%%%%%%%%%%%%%%%%%%%%%%%%%%%%%%%%%%%%%%%%%
\end{document}
